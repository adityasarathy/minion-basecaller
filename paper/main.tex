\documentclass[times, utf8, seminar]{fer}
\usepackage{booktabs}

\usepackage{mathtools, amsmath,amsfonts,amssymb, amsthm}
\usepackage{acronym}
\usepackage{geometry}
\usepackage{hyperref}
\usepackage{float}
\usepackage{graphicx}
\graphicspath{ {Figures/} }

\usepackage{tikz}
\usetikzlibrary{shapes,arrows}
\begin{document}
\theoremstyle{definition}
\newtheorem{definition}{Definition}[section]

\voditelj{Mile Sikić}
\title{CNNNano --- MinION  basecaller pogonjen dubokim ucenjem}
\engtitle{CNNNano --- MinION deep learning powered basecaller}
\author{Neven Miculinić}

\maketitle
\tableofcontents

\begin{sazetak}
TODO: ovo je pre-alpha draft...bude se vec osvjezilo


\kljucnerijeci{planiranje nabave, stohastička potražnja, optimalna količina nabave, Model dinamičke ekonomske količine nabave}
\end{sazetak}

\begin{abstract}
    The MinION device by Oxford Nanopore is the first portable sequencing device. MinION is able to produce very long reads (reads over 100 kBp were reported), however it suffers from high sequencing error rate. In this paper, we show that the error rate can be reduced by improving the base calling process using residual convolutional neural networks(CNN) combined with Connectionist Temporal Classification (CTC).
\keywords{procurement planning, dynamic lot sizing model, stochastic demands}
\end{abstract}

\chapter{Introduction}

In this paper, we introduce the base caller for the MinION nanopore sequencing platform (Mikheyev and Tin, 2014).
The MinION device by Oxford Nanopore, weighing only 90 grams, is currently the smallest high-throughput DNA sequencer. Thanks to its low capital costs, small size and the possibility of analyzing the data in real time as they are produced, MinION is very promising for clinical applications, such as monitoring infectious disease outbreaks (Judge et al., 2015; Quick et al., 2015, 2016), and characterizing structural variants in cancer (Norris et al., 2016). Although MinION is able to produce long reads, they have a high sequencing error rate. In this paper, we show that this error rate can be reduced by replacing the default base caller provided by the manufacturer with a properly trained neural network model.

In the MinION device, single-stranded DNA fragments move through nanopores, which causes drops in the electric current. The electric current is measured at each pore several thousand times per second. The electric current depends mostly on the context of several DNA bases passing through the pore at the time of measurement. As the DNA moves through the pore, the context shifts and the electric current changes.

A MinION device typically yields reads several thousand bases long; reads as long as 100,000 bp have been reported. 1D read have usually error rate about   30\%. After parameter training, base calling can be performed by running the Viterbi algorithm, which will result in the sequence of states with the highest likelihood.

\chapter{Related work}

Official MinION basecaller is

Note that neural networks were previously used for base calling Sanger sequencing reads (Tibbetts et al., 1994; Mohammed et al., 2013), though the nature of MinION data is rather different. Several tools for processing nanopore sequencing data were already published, including read mappers (Jain et al., 2015; Sovic et al., 2015), and error correction tools using short Illumina reads (Goodwin et al., 2015). Most closes related to our work are Nanopolish (Loman et al., 2015b) and PoreSeq (Szalay and Golovchenko, 2015). Both of these tools create a consensus sequence by combining information from multiple overlapping reads, considering not only the final base calls from Metrichor, but also the sequence of events. They analyze the events by hidden Markov models with emission probabilities provided by Metrichor. In contrast, our base caller does not require read overlaps, it processes reads individually and

\chapter{Model description}
Lorem ipsum dolor sit amet, consectetur adipisicing elit, sed do eiusmod tempor incididunt ut labore et dolore magna aliqua. Ut enim ad minim veniam, quis nostrud exercitation ullamco laboris nisi ut aliquip ex ea commodo consequat. Duis aute irure dolor in reprehenderit in voluptate velit esse cillum dolore eu fugiat nulla pariatur. Excepteur sint occaecat cupidatat non proident, sunt in culpa qui officia deserunt mollit anim id est laborum.
\chapter{Experimental results}
Lorem ipsum dolor sit amet, consectetur adipisicing elit, sed do eiusmod tempor incididunt ut labore et dolore magna aliqua. Ut enim ad minim veniam, quis nostrud exercitation ullamco laboris nisi ut aliquip ex ea commodo consequat. Duis aute irure dolor in reprehenderit in voluptate velit esse cillum dolore eu fugiat nulla pariatur. Excepteur sint occaecat cupidatat non proident, sunt in culpa qui officia deserunt mollit anim id est laborum.
\chapter{Conclusion and further work}
Lorem ipsum dolor sit amet, consectetur adipisicing elit, sed do eiusmod tempor incididunt ut labore et dolore magna aliqua. Ut enim ad minim veniam, quis nostrud exercitation ullamco laboris nisi ut aliquip ex ea commodo consequat. Duis aute irure dolor in reprehenderit in voluptate velit esse cillum dolore eu fugiat nulla pariatur. Excepteur sint occaecat cupidatat non proident, sunt in culpa qui officia deserunt mollit anim id est laborum.
% \cite{deepnano}
% \bibliographystyle{fer}
% \bibliographystyle{abbrv}
% \bibliography{refs}

\end{document}
